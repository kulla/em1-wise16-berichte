\documentclass[a4paper]{article}

\usepackage[utf8]{inputenc}
\usepackage{amsmath}
\usepackage{geometry}
\usepackage{ngerman}
\usepackage{parskip}
\usepackage{amssymb}

\newcommand*{\R}{\mathbb R}
\newcommand*{\N}{\mathbb N}
\newcommand*{\Z}{\mathbb Z}
\newcommand*{\Q}{\mathbb Q}

\title{Bericht Übungsblatt 8}
\date{}
\author{}

\begin{document}
\maketitle

\section{Aufgaben zur Integration}

Folgende werden zusammenfassend behandelt, da sie ähnliche Lösungswege haben.

\subsection{Aufgabe 2.5.2}

Sei $E$ die achsenparallele Ellipse mit den Halbachsen $a$,$b>0$, d.h.

\begin{align}
  E=\left\{(x,y)\in\R^2: \left(\frac xa\right)^2+\left(\frac yb\right)^2=1\right\}
\end{align}

Skizzieren Sie $E$ und bestimmen Sie die Fläche von $E$ in Abhängigkeit von den Parametern $a$ und $b$ mit Hilfe von Integration.

\subsection{Aufgabe 2.5.4}

Bestimmen Sie durch Andwendung der Integrationsregeln:

\begin{enumerate}
  \item $\int \frac{\log(x)}{x} \mathrm dx$
  \item $\int x\log(x) \mathrm dx$
  \item $\int \log^2(x) \mathrm dx$
  \item $\int \log^3(x) \mathrm dx$
  \item $\int \sqrt{x} \log(x) \mathrm dx$
  \item $\int \log\left(\sqrt x\right) \mathrm dx$
\end{enumerate}

\subsection{Aufgabe 2.5.6}

Seien $a,b\in\R$ mit $b>0$. Bestimmen Sie durch Anwendung der Integrationsregeln:

\begin{align}
  \int \frac{1-ae^x}{1+be^x} \mathrm dx
\end{align}

\subsection{Anmerkungen zu den Aufgaben}

Die Aufgaben zur Integration wurden sehr gut gelöst. Auch nutzen die Studierenden in ihren Lösungen viele verschiedene Wege. Dies deutet darauf hin, dass viele die Aufgaben eigenständig bearbeitet haben. Es gab nur wenige Probleme, die dabei auftraten. Diese waren:

\begin{itemize}
  \item Anstelle von $\frac{\mathrm d}{\mathrm dt} f(t)$ wurde manchmal $f(t) \mathrm dt$ geschrieben.
  \item An einzelnen Stellen gab es Vorzeichenfehler.
  \item Anstelle der Stammfunktion wurde selten die Ableitung berechnet.
  \item Manchmal wurde der Äquivalenzpfeil anstelle eines Gleichheitszeichens eingesetzt.
  \item Bei Termumformungen war es oft nicht ersichtlich, was ein Kommentar und was der nächste Term der Termumformung ist.
  \item Manchmal wurde in der ersten Aufgaben anstelle des bestimmten Integrals das unbestimmte Integral ausgerechnet.
  \item Bei Berechnung des bestimmten Integrals mit Hilfe der Substitution wurde manchmal eine Rücksubstitution durchgeführt, obwohl dies unnötig ist.
\end{itemize}

\section{Aufgabe 2.5.11}

\subsection{Aufgabe}

Sei $f:[a,b]\to[0,\infty)$ streng monoton wachsend und stetig differenzierbar. Weiter seien $c=f(a)$ und $d=f(b)$.

\begin{enumerate}
  \item Finden Sie mit Hilfe eines Diagramms und der geometrischen Interpretation des Integrals als Flächeninhalt eine Formel für $\int_c^d f^{-1}(x)\mathrm dx$ in Abhängigkeit von $\int_a^b f(x) \mathrm dx$.
  \item Beweisen Sie Ihre Formel mit Hilfe der Integrationsregeln.
\end{enumerate}

\subsection{Anmerkungen}

Diese Aufgabe wurde in der Regel richtig gelöst. Das Diagramm wie auch die zu beweisende Formel war in der Regel richtig. Einige brauchten aber Unterstützung via WhatsApp (Beantwortung von Verständnisfragen). Auch der Beweis war in der Regel richtig.

\end{document}

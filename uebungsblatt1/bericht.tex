\documentclass[a4paper]{article}

\usepackage[utf8]{inputenc}
\usepackage{amsmath}
\usepackage{geometry}
\usepackage{ngerman}
\usepackage{parskip}
\usepackage{amssymb}

\newcommand*{\R}{\mathbb R}
\newcommand*{\N}{\mathbb N}
\newcommand*{\Z}{\mathbb Z}
\newcommand*{\Q}{\mathbb Q}
\newcommand*{\dx}{\,\mathrm{d}x}

\title{Übungsblatt 1}
\date{}
\author{}

\begin{document}
\maketitle

\section{Aufgabe 1.1.1}

\begin{enumerate}
  \item Bestimmen Sie die Steigungsform einer Geraden $g:\R\to\R$ durch zwei Punkte $(x_1,y_1)$ und $(x_2,y_2)$, $x_1\neq x_2$ in den Entwicklungspunkten $x_1$ und $x_2$.

  \item Nehmen Sie nun an, dass $y_1$ und $y_2$ die Werte einer Funktion $f:\R\to\R$ an den Stellen $x_1$ und $x_2$ sind. Schreiben Sie die Steigungsform für diese Situation auf und erstellen Sie ein zugehöriges Diagramm.
\end{enumerate}

\section{Aufgabe 1.1.3}

Seien $g,h:\R\to\R$ Geraden. Weiter sei $f=h\circ g$, d.h. es gilt $f(x)=h(g(x))$ für alle $x\in\R$. Zeigen Sie, dass $f$ eine Gerade ist. Finden Sie eine Bedingung für $f(0)=0$.

\end{document}

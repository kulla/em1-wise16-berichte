\documentclass[a4paper]{article}

\usepackage[utf8]{inputenc}
\usepackage{amsmath}
\usepackage{geometry}
\usepackage{ngerman}
\usepackage{parskip}
\usepackage{amssymb}

\newcommand*{\R}{\mathbb R}
\newcommand*{\N}{\mathbb N}
\newcommand*{\Z}{\mathbb Z}
\newcommand*{\Q}{\mathbb Q}

\title{Hinweise zum Übungsbetrieb}
\date{}

\begin{document}
\maketitle

\begin{itemize}
\item Abgabe an stephan.kulla@tum.de
\item Den Notenbonus erhaltet ihr, wenn ihr mindestens 70\% der Übungsaufgaben (und nicht der Übungsblätter) sinnvoll bearbeitet. Während des Videofeedbacks sage ich euch, ob ihr eine Übungsaufgabe sinnvoll bearbeitet habt oder nicht. Gerne könnt ihr mir eine E-Mail schicken, wenn ihr die Anzahl der von euch sinnvoll erarbeiteten Aufgaben erfahren wollt.
\item Bitte schreibt euren Namen auf die abgegebenen Lösungen.
\item Ihr könnt gerne zusammen arbeiten. Wenn ihr dies macht, dann gebt bitte nur eine Lösung ab und schreibt die Namen aller Beteiligten auf die Lösung.
\item Bitte achtet bei der Abgabe eurer Lösung darauf, dass alle Fotos / Scans dieselbe Orientierung haben.
\item Bitte erstellt keine Schwarz-Weiß-Scans, sondern nur Scans in Graustufen.
\item Es ist egal, ob ihr mir die Lösungen über eure TUM-Adresse oder euren privaten Accounts zusendet. In der Regel schicke ich die Zugriffslinks an die Adresse, über die ihr mir eure Lösungen zugeschickt habt.
\item Bei Android-Geräten scheint es Probleme mit dem Abspielen der Videos zu geben (schwarzes Bild). Wenn ihr eure Videos mit Android-Geräten abspielen wollt, dann schickt mir eine Mail. Ich sende euch dann das Video in einem anderen Format.
\item Am Mittwoch habe ich vor der Übung keine Zeit, euch Feedback zu geben. Wenn ihr mir eure Lösungen bis Dienstag 16/17 Uhr zuschickt, dann schaue ich, dass ich diese noch am selben Tag korrigiere, so dass ihr noch am Mittwoch mir eine Verbesserung zuschicken könnt. Beim Korrigieren gehe ich im Übrigen chronologisch zum Eingang der E-Mail vor.
\end{itemize}

\end{document}

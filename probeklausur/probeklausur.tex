\documentclass[a4paper]{article}

\usepackage[utf8]{inputenc}
\usepackage{amsmath}
\usepackage{geometry}
\usepackage{ngerman}
\usepackage{parskip}
\usepackage{amssymb}

\newcommand*{\R}{\mathbb R}
\newcommand*{\N}{\mathbb N}
\newcommand*{\Z}{\mathbb Z}
\newcommand*{\Q}{\mathbb Q}
\newcommand*{\dx}{\,\mathrm{d}x}

\title{Probeklausur 11.01.2017}
\date{}
\author{}

\begin{document}
\maketitle

\section{Aufgabe 1 (3 Punkte)}

Ergänzen bzw. definieren Sie:

\begin{enumerate}
  \item Potenzreihenentwicklung des Logarithmus im Punkt 1 % Klausur 15/16
  \item Arkussinus-Funktion % Probeklausur 15/16
  \item Taylor-Polynom dritter Ordnung von $f:\R\to\R$ im Entwicklungspunkt 1 % Probeklausur 15/16
\end{enumerate}

\section{Aufgabe 2 (6 Punkte)}

Geben Sie eine exakte und vollständige Formulierung der folgenden Sätze / Axiome an:

\begin{enumerate}
  \item Quotientenkriterium für eine unendliche Reihe $\sum_{n=0}^\infty x_n$ in $\R$. % Wiederholungsklausur 15/16
  \item Hauptsatz der Integral- und Differentialrechnung: Existenz von Stammfunktionen % Klausur 15/16
  \item Archimedisches Axiom
\end{enumerate}

\section{Aufgabe 3 (5 Punkte)} % Klausur 15/16

\begin{enumerate}
  \item Zeigen Sie, dass $\frac{\mathrm d}{\mathrm dx} \operatorname{arccos}(x) = -\frac{1}{\sqrt{1-x^2}}$. Verwendete nichtelemtare trogonometrische Identitäten sind zu beweisen. % 2 Punkte
  \item Zeigen Sie, dass $\int_0^\pi \sqrt{1-\cos(x)} \dx = 2\sqrt 2$. % 3 Punkte
\end{enumerate}

\section{Aufgabe 4 (12 Punkte)}

Sei $f:\R\to\R$ mit

\begin{align}
  f(x) = \begin{cases} \sin\left(\frac 1x\right) & x \neq 0 \\ 0 & x = 0 \end{cases}
\end{align}

\begin{enumerate}
  \item Skizzieren Sie den Graphen von $f$ qualitativ.
  \item Bestimmen Sie die Menge aller Argumente $x$ mit $f(x)=1$.
  \item Beweisen Sie, dass $f$ an der Stelle $p=0$ unstetig ist.
  \item Illustrieren Sie ihre Beweisidee zur Unstetigkeit von $f$ durch eine Visualisierung.
\end{enumerate}

\end{document}

\documentclass[a4paper]{article}

\usepackage[utf8]{inputenc}
\usepackage{amsmath}
\usepackage{geometry}
\usepackage{ngerman}
\usepackage{parskip}
\usepackage{amssymb}

\newcommand*{\R}{\mathbb R}
\newcommand*{\N}{\mathbb N}
\newcommand*{\Z}{\mathbb Z}
\newcommand*{\Q}{\mathbb Q}

\title{Übungsblatt 5}
\date{}
\author{}

\begin{document}
\maketitle

\section{Aufgabe 6.6}

\subsection{Aufgabenstellung}

Zeigen Sie, dass für alle $x\in\R$ unter der Voraussetzung der Definiertheit die folgenden Halbierungsformeln gelten:

\begin{enumerate}
  \item $\cos^2\left(\frac x2\right) = \frac{\cos(x)+1}{2}$
  \item $\sin^2\left(\frac x2\right) = \frac{1-\cos(x)}{2}$
  \item $\tan\left(\frac x2\right) = \frac{1-\cos(x)}{\sin(x)} = \frac{\sin(x)}{1+\cos(x)}$
\end{enumerate}

Welche Vorzeichen sind beim Wurzelziehen in (a) und (b) abhängig von $x$ zu wählen?

\subsection{Anmerkungen}

Diese Aufgabe bereitete wenige Schwierigkeiten. Problematisch war hingegen die Zusatzaufgabe, in der die Vorzeichen von $\cos(x/2)$ bzw. $\sin(x/s)$ bestimmt werden sollten. Rückmeldungen deuten darauf hin, dass einige diese Zusatzaufgabe nicht richtig verstanden hatten. Die, die diese Zusatzaufgabe gelöst hatten, haben zum Teil die Vorzeichen von $\sin(x)$ bzw. $\cos(x)$ wiedergegeben. Andere haben die Vorzeichen nur für ein Intervall der Länge $4\pi$ bestimmt, jedoch nicht für beliebige $x\in\R$.

In der Teilaufgabe (c) wurde manchmal die Aufgabe so gelöst, dass die Wurzel bei den Aufgaben (a) und (b) gezogen wurde. Hier wurden aber nicht die Vorzeichen betrachtet. Es wurden nur die nichtnegativen Wurzeln benutzt.q Andere haben dies vermieden, indem sie die Verdopplungsformeln verwendeten.

\section{Aufgabe 6.10}

\subsection{Aufgabenstellung}

Sei $f:\R\to\R$. Zeigen Sie, dass die folgenden Aussagen äquivalent sind:

\begin{enumerate}
  \item Es gibt $c,t_0\in\R$ mit $f(t)=c\cos(t+t_0)$ für alle $t\in\R$.
  \item Es gibt $a,b\in\R$ mit $f(t)=a\cos(t)+b\sin(t)$ für alle $t\in\R$.
\end{enumerate}

\subsection{Anmerkungen}

Diese Aufgabe hat die meisten Schwierigkeiten bereitet. Es war wohl vielen nicht klar, wie eine Beweis hier geführt werden muss. In den abgegebenen Lösungen wurde oft der Beweis (a) nach (b) geführt (ohne dass dies explizit genannt wurde). Jedoch fehlte die Implikation (b) nach (a). Erste Rückmeldungen lassen darauf schließen, dass den Studierenden auch nicht klar ist, wie sie in dieser Aufgabe auf eine Lösung kommen. Insofern lohnt es sich, diese Aufgabe bei den Übungsstunden detailliert zu erklären.

Nach einem entsprechenden Hinweisvideo wurden die Lösungen besser, jedoch war oft der Beweisschritt (b) nach (a) fehlerhaft (während der Beweisschritt (a) nach (b) oft richtig war). Hier wurde zum Beispiel $a=c \cos t_0$ und $b=-c\sin t_0$ gesetzt, obwohl hier $c$ und $t_0$ mit Hilfe von $a$ und $b$ definiert werden müssen.

Im Beweisschritt $\text{(b)} \implies \text{(a)}$ ist eine Fallunterscheidung in $a=0$ und $a\neq 0$ notwendig. Diese Fallunterscheidung wurde nur selten durchgeführt.

\section{Aufgabe 6.12}

\subsection{Aufgabenstellung}

Illustrieren Sie die Formeln

\begin{enumerate}
  \item $\sin(\operatorname{arccos}(x)) = \cos(\operatorname{arcsin}(x)) = \sqrt{1-x^2}$
  \item $\tan(\operatorname{arccot}(x)) = \cot(\operatorname{arctan}(x)) = \frac 1x$
\end{enumerate}

geometrisch mit Hilfe des Einheitskreises.

\subsection{Anmerkungen}

Diese Aufgabe wurde in der Regel gut gelöst. Schwierigkeiten hat die zweite Teilaufgabe bereitet, da zum Teil nicht bekannt war, wo der Tangens und der Kotangens im Einheitskreis zu finden ist. Wenige haben $x$ als Winkel interpretiert. Auch hat es Schwierigkeiten bereitet, dass $x$ in den Formeln $\sin(\operatorname{arcsin}(x))$ bzw. $\cos(\operatorname{arcsin}(x))$ verschiedene Strecken meint. Dasselbe Problem trat auch in den Formeln $\cot(\operatorname{arccot}(x))$ und $\tan(\operatorname{arctan}(x))$ auf.

\section{Aufgabe 7.4}

\subsection{Aufgabenstellung}

Zeigen Sie, dass $H=\{(x,y)\in\R^2: x^2-y^2=1,x\ge 1\}$ (rechter Ast der Einheitshyperbel) und der Graph der Funktion $f:\R^{+}\to\R$ mit $f(x)=\frac{1}{2x}$ für alle $x>0$ durch eine Drehung um $\frac \pi4$ gegen bzw. im Uhrzeigersinn auseinander hervorgehen. Erstellen Sie zudem ein Diagramm zur Illustration.

\subsection{Anmerkungen}

Diese Aufgabe wurde weniger oft gelöst. Wenn es aber eine Lösung gab, dann waren die Lösungen in den meisten Fällen gut. Nur die Illustrationen wurden zum Teil falsch angefertigt. So wurde zum Teil eine Drehung um $\tfrac \pi2$ anstatt eine Drehung um $\tfrac \pi4$ eingezeichnet. Auch war manchmal der Graph der Funktion $f$ falsch.

\end{document}
